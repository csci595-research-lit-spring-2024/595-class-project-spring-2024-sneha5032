\chapter{Literature Review}
\label{ch:lit_rev} %Label of the chapter lit rev. The key ``ch:lit_rev'' can be used with command \ref{ch:lit_rev} to refer this Chapter.
Recognizing the imperative for automated solutions to distinguish genuine disaster-related 
tweets from metaphorical or unrelated content, the research gets crucial reference, aiding in 
exploring effective approaches and refining the accuracy of disaster-related tweet prediction 
models from this Kaggle competition \cite{nlp-getting-started} \cite{nlp-getting-started2}. 
With widespread smartphone use, Twitter becomes a primary source for disaster-related 
information. Utilizing a dataset of 10,000 hand-classified tweets, the goal is to employ 
machine learning models for binary classification and NLP to predict tweets genuinely related 
to disasters. 

% PLEAE CHANGE THE TITLE of this section
\section{Related Work} 
In the research \cite{chanda2021efficacy} extensive evaluation of Deep Learning methods for 
classifying disaster-related tweets. Identification of preprocessing steps, emphasizing named 
entity substitution.Performance analysis of custom neural networks and Transformer 
models.Emphasis on practical application potential for automatic disaster detection. BERT is 
used in \cite{deb2022comparative} where exploration of Twitter's real-time data for disaster 
identification and challenges in manual data processing due to volume is elaborated. Evaluation 
of BERT embeddings' superiority in disaster prediction, compared to traditional word 
embeddings. Discussion on opportunities and challenges of BERT embeddings in Twitter sentiment 
analysis. Detailed analysis for various algorithms are explored in 
\cite{fontalis2023comparative} Theoretical basis of various ML algorithms for tweet analysis 
(BNB, MNB, LR, KNN, DT, RF).Process flow from dataset import to model training, emphasizing the 
importance of Exploratory Data Analysis (EDA).
Selection of ML models based on suitability, using Wordclouds for identifying relevant 
words.Explanation of Bayesian algorithms, logistic regression, decision tree, and random forest 
usage.\cite{iparraguirre2023classification}Exploration of BERT embeddings' effectiveness in 
disaster prediction on social media. Overview of challenges in manual disaster identification 
due to data volume. Application of different word embeddings (BOW, context-free, contextual) in 
disaster prediction models.
Utilization of embeddings in both traditional ML methods and neural network-based models.In 
\cite{saddam2023sentiment}Studies introduce normalization processes for words with similar 
meanings. Stemming, using the Indonesian literary library in Python, maximizes text processing 
efficieny. Machine learning involves labeling real opinions, crucial for SVM model training.
SVM models, especially for multiclass classification, are discussed for sentiment analysis. K-
Fold Cross Validation ensures robust testing, evaluating accuracy, precision, recall, and F-
score. Confusion matrices aid in a comprehensive understanding of model performance.

% A possible section of you chapter
\section{Critique of the review} % Use this section title or choose a betterone
Existing research demonstrates the effectiveness of advanced techniques such as deep learning 
models and contextual embeddings for disaster-related sentiment analysis. Critique emphasizes 
the need for continuous improvement in addressing challenges related to bias mitigation and 
contextual understanding.

% Pleae use this section
\section{Summary} 
The literature review underscores the significance of sophisticated preprocessing techniques, 
advanced text processing, and the application of machine learning models for disaster-related 
sentiment analysis on Twitter. Key findings highlight the potential of deep learning and 
contextual embeddings in achieving accurate disaster detection, with practical implications for 
real-world applications. Continuous improvement is encouraged to address existing challenges 
and further enhance the reliability of sentiment analysis systems.
