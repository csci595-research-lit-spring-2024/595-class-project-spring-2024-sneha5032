\chapter{Introduction}
\label{ch:into} % This how you label a chapter and the key (e.g., ch:into) will be used to refer this chapter ``Introduction'' later in the report. 
% the key ``ch:into'' can be used with command \ref{ch:intor} to refere this Chapter.

Twitter has emerged as a ubiquitous platform for event reporting, facilitated by the widespread use of smartphones. This dynamic environment offers unparalleled opportunities for immediate and decentralized communication, especially during critical events such as disasters. The advent of this digital era has underscored the pressing need for effective crisis communication strategies to harness the potential of Twitter as a valuable tool for situational awareness and emergency response.

However, amidst the wealth of real-time data flowing through Twitter feeds, a significant challenge arises in the accurate identification and differentiation of metaphorical expressions from authentic crisis-related information within tweets. Metaphors, while a powerful linguistic tool for expression, often introduce ambiguity and complexity, posing a considerable hurdle in the quest for reliable crisis detection. The inherent nature of metaphorical language requires a nuanced understanding that transcends traditional analytical approaches, demanding innovative solutions to decipher the true intent behind tweets during critical events.

This research embarks on the journey to address this multifaceted challenge by delving into the intricacies of Twitter communication during crises. The aim is to develop a sophisticated machine learning framework capable of distinguishing between metaphorical language and genuine crisis-related information. Leveraging the prevalence of smartphones and the instantaneous nature of Twitter reporting, this study seeks to contribute to the advancement of crisis communication strategies, fostering a more effective and accurate response to emergencies in the digital age.

%%%%%%%%%%%%%%%%%%%%%%%%%%%%%%%%%%%%%%%%%%%%%%%%%%%%%%%%%%%%%%%%%%%%%%%%%%%%%%%%%%%
\section{Background}
\label{sec:into_back}
The motivation stems from the critical need for effective crisis communication strategies in utilizing Twitter as a valuable tool for situational awareness and emergency response.

The central challenge lies in accurately discerning metaphorical expressions from genuine crisis-related information within tweets. Metaphors, while powerful for expression, introduce ambiguity, posing a significant hurdle to reliable crisis detection. This project addresses this challenge through the development of a sophisticated machine learning framework, drawing on established classification algorithms with intentional omission of specific names for flexibility. Hyperparameter tuning and model selection are explored for optimization. Concurrently, Natural Language Processing (NLP) models capture contextual nuances to enhance metaphorical language understanding.


%%%%%%%%%%%%%%%%%%%%%%%%%%%%%%%%%%%%%%%%%%%%%%%%%%%%%%%%%%%%%%%%%%%%%%%%%%%%%%%%%%%
\section{Problem statement}
\label{sec:intro_prob_art}

The significant challenge in the realm of disaster monitoring involves distinguishing disaster-related tweets from general Twitter content. This study aims to develop a machine learning algorithm capable of addressing this challenge and accurately determining if a tweet is genuinely connected to a crisis.

%%%%%%%%%%%%%%%%%%%%%%%%%%%%%%%%%%%%%%%%%%%%%%%%%%%%%%%%%%%%%%%%%%%%%%%%%%%%%%%%%%%
\section{Aims and objectives}
\label{sec:intro_aims_obj} 

 
  \textbf{Aims:}
The primary aim of this project is to enhance the field of disaster monitoring on Twitter by developing a sophisticated machine learning framework. The goal is to accurately distinguish tweets related to disasters from the broader spectrum of general content on the platform. Through this endeavor, we seek to contribute to the improvement of crisis communication strategies in the digital age.

\textbf{Objectives:} Implement data exploration and preprocessing techniques to ensure the dataset is prepared for training and evaluation.
Explore and apply various machine learning classification algorithms for the effective categorization of tweets, emphasizing the optimization of hyperparameters and model selection.

Incorporate Natural Language Processing (NLP) models to capture contextual nuances, enhancing the understanding of metaphorical language within tweets.
Evaluate the performance of the developed framework using rigorous metrics to ensure accuracy and reliability in distinguishing disaster-related tweets.
Provide insights and recommendations for advancing crisis communication strategies based on the project outcomes.


%%%%%%%%%%%%%%%%%%%%%%%%%%%%%%%%%%%%%%%%%%%%%%%%%%%%%%%%%%%%%%%%%%%%%%%%%%%%%%%%%%%
\section{Solution approach}
\label{sec:intro_sol} % label of Org section
The solution approach consists of distinct stages, including data collection, exploration, preprocessing, and the implementation of algorithms for classification and NLP.

\subsection{Data Collection, Exploration, and Preprocessing}
\label{sec:intro_some_sub1}
\textbf{Data Exploration:} Comprehensive exploration is conducted to understand the characteristics of the dataset, including the distribution of metaphorical expressions and crisis-related content.
\textbf{Preprocessing:} Textual data undergoes preprocessing, including tokenization, stemming, and handling of special characters, to prepare it for subsequent stages. 

\subsection{Implementation of Algorithms}
\label{sec:intro_some_sub2}
 Established machine learning classification algorithms are implemented to effectively classify tweets. Hyperparameter tuning and model selection are explored to optimize performance.
NLP models are implemented to capture contextual nuances and improve the understanding of metaphorical language in tweets.


%%%%%%%%%%%%%%%%%%%%%%%%%%%%%%%%%%%%%%%%%%%%%%%%%%%%%%%%%%%%%%%%%%%%%%%%%%%%%%%%%%%
\section{Summary of contributions and achievements} %  use this section 
\label{sec:intro_sum_results} % label of summary of results
This research contributes a sophisticated machine learning framework capable of distinguishing metaphorical language from genuine crisis-related information on Twitter. Achievements include the development of a robust algorithm, leveraging a hand-classified dataset for effective model training.


%%%%%%%%%%%%%%%%%%%%%%%%%%%%%%%%%%%%%%%%%%%%%%%%%%%%%%%%%%%%%%%%%%%%%%%%%%%%%%%%%%%
\section{Organization of the report} %  use this section
\label{sec:intro_org} % label of Org section
The report follows a structured format, exploring background, problem statement, solution approach, detailed methodologies, results, discussions, and conclusions.
