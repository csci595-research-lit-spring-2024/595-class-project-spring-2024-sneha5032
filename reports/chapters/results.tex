\chapter{Results}

The training and evaluation of the logistic regression model yielded promising results, with a training accuracy of 96 percent and a validation accuracy of 75 percent. This indicates that the model performed well in learning from the training data but showed a slight decrease in accuracy when tested on unseen validation data. On the other hand, the DistilBERT model achieved a training accuracy of 93 percent and a validation accuracy of 77 percent, showcasing its capability in handling complex textual data and performing consistently across both training and validation sets.

Further optimization using DistilBERT with AUTOTUNE led to significant improvements, with a training accuracy of 98percent and a validation accuracy of 78 percent. The dynamic optimization provided by AUTOTUNE played a crucial role in enhancing the model's generalization ability, allowing it to maintain high accuracy on unseen data while leveraging efficient resource utilization.
\label{ch:results}
\begin{table}[ht]
\caption{Performance Comparison of Different Models}
\begin{tabular}{lllll}

model                           & Traning score & Validation score & Time to train model &  \\
Logistic regression             & 96            & 75               & 1s                  &  \\
Distilbert with parameter table & 93            & 77               & 4m                  &  \\
Distilbert with Autotune        & 98            & 78               & 7m                  & 
\end{tabular}
\end{table}
